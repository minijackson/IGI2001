\documentclass{report}

\usepackage[utf8]{inputenc}
\usepackage[T1]{fontenc}
\usepackage[french]{babel}
\usepackage[]{amsmath}
\usepackage{amsfonts}
\usepackage{mathabx}
\usepackage{fullpage}
\usepackage[usenames,dvipsnames]{color}
\usepackage{listingsutf8}
\usepackage[hidelinks]{hyperref}

\definecolor{gray}{gray}{0.70}

\lstset{%
	tabsize=4,
	basicstyle=\small\ttfamily,
	numberstyle=\scriptsize\ttfamily,
	texcl=true,
	extendedchars=true,
	inputencoding=utf8/latin1
}

\lstdefinestyle{prog}{%
	caption=Programme,
	numbers=left,
	frame=L,
	language=c,
	tabsize=4,
	basicstyle=\small\ttfamily,
	numberstyle=\scriptsize\ttfamily,
	keywordstyle=\color{BlueViolet},
	stringstyle=\color{Green},
	commentstyle=\color{gray},
	showstringspaces=false
}

\lstdefinestyle{output}{%
	caption=Résultat,
	frame=single,
	belowcaptionskip=1\baselineskip,
	breaklines=true
}

\lstdefinestyle{outfile}{%
	caption=Fichier de sortie,
	frame=shadowbox,
	rulesepcolor=\color{gray},
	belowcaptionskip=1\baselineskip,
	breaklines=true
}

\author{Rémi \textsc{Nicole}}
\title{Making of Hangman2}
\date{}

\begin{document}

\maketitle

\chapter{The Game}

\section{Main game}

\subsection{Picking a random word}

\paragraph{} In order to pick a random word in a file consisting of a single
word per line, the file is first scanned to count the number of files ($n$),
then a random number $k$ is picked in the $\ldbrack1;n\rdbrack$ interval.
After that, the $k^{th}$ word of the file is picked and returned as a
\lstinline[style=prog]|static| string (prevent the use of
\lstinline[style=prog]|malloc|).

\subsection{Game word and hidden word}

\paragraph{} Two sting variables are instantiated for the random word: one
which is the random word (game word) and another which is originally composed
of underscores (\texttt{\_}) only (hidden word). The second variable will be
the text displayed to the player and will serve to check if the player won: if
this string does not contains any underscores, then the player won. Each time
the player inputs a character, the inputted character is searched through the
game word and if found, any occurrences of this character will replace the
corresponding underscore(s) in the hidden word. If not found, the player will
lose one attempt.

\section{Details}

\subsection{Console ``clear''}

\paragraph{}

\subsection{Invalid characters}

\paragraph{} Each time the player inputs a character, the character inputted is
checked through the function \lstinline[style=prog]|validLetter| and display
\texttt{Invalid character} if the character inputted is not a letter. If the
letter inputted is a \lstinline[style=prog]|'\n'|, the console curser goes up
one line more.

\end{document}
% vim: spell
