\documentclass{report}

\usepackage[utf8]{inputenc}
\usepackage[T1]{fontenc}
\usepackage[french]{babel}
\usepackage{amsmath}
\usepackage{amsfonts}
\usepackage{mathabx}
\usepackage{fullpage}
\usepackage{dirtree}
\usepackage[usenames,dvipsnames]{color}
\usepackage{listingsutf8}
\usepackage[hidelinks]{hyperref}

\definecolor{gray}{gray}{0.70}

\lstset{%
	tabsize=4,
	basicstyle=\small\ttfamily,
	numberstyle=\scriptsize\ttfamily,
	texcl=true,
	extendedchars=true,
	inputencoding=utf8/latin1
}

\lstdefinestyle{prog}{%
	caption=Programme,
	numbers=left,
	frame=L,
	language=c,
	tabsize=4,
	basicstyle=\small\ttfamily,
	numberstyle=\scriptsize\ttfamily,
	keywordstyle=\color{BlueViolet},
	stringstyle=\color{Green},
	commentstyle=\color{gray},
	showstringspaces=false
}

\lstdefinestyle{output}{%
	caption=Résultat,
	frame=single,
	belowcaptionskip=1\baselineskip,
	breaklines=true
}

\lstdefinestyle{outfile}{%
	caption=Fichier de sortie,
	frame=shadowbox,
	rulesepcolor=\color{gray},
	belowcaptionskip=1\baselineskip,
	breaklines=true
}

\author{Rémi \textsc{Nicole}}
\title{Making of Hangman2}
\date{}

\begin{document}

\maketitle

\chapter{Directory tree of the project}

\dirtree{%
 .1 [root].
   .2 configure.ac\DTcomment{Autoconf main configration file}.
   .2 Makefile.am.
   .2 src.
     .3 Makefile.am.
	 .3 hangman2\DTcomment{Executable}.
	 .3 hangman2.c\DTcomment{Main source file}.
	 .3 config.h\DTcomment{\lstinline[style=prog]|##define| bundle generated by autoconf}.
	 .3 utils.
	   .4 dict.c.
	   .4 dict.h.
	   .4 words.c.
	   .4 words.h.
	   .4 pics.c.
	   .4 pics.h.
	   .4 inc.h\DTcomment{Common \lstinline[style=prog]|##include| and \lstinline[style=prog]|##define|}.
     .3 pics\DTcomment{Default pictures directory}.
	   .4 1.txt.
	   .4 2.txt.
	   .4 3.txt.
	   .4 4.txt.
	   .4 5.txt.
	   .4 6.txt.
	   .4 7.txt.
	   .4 8.txt.
	   .4 9.txt.
	   .4 10.txt.
   .2 doc\DTcomment{Documentation directory}.
     .3 Makefile.am.
	 .3 report.tex.
	 .3 report.pdf.
   .2 NEWS.
   .2 AUTHORS.
   .2 COPYING.
   .2 INSTALL.
   .2 LICENSE.
   .2 README.
   .2 THANKS.
   .2 ChangeLog.
}

\chapter{The Game}

\section{Main game}

\subsection{Picking a random word}

\paragraph{} In order to pick a random word in a file consisting of a single
word per line, the file is first scanned to count the number of files ($n$),
then a random number $k$ is picked in the $\ldbrack1;n\rdbrack$ interval.
After that, the $k^{th}$ word of the file is picked and returned.

\subsection{Game word and hidden word}

\paragraph{} Two sting variables are instantiated for the random word: one
which is the random word (game word) and another which is originally composed
of underscores (\texttt{\_}) only (hidden word). The second variable will be
the text displayed to the player and will serve to check if the player won: if
this string does not contains any underscores, then the player won. Each time
the player inputs a character, the inputted character is searched through the
game word and if found, any occurrences of this character will replace the
corresponding underscore(s) in the hidden word. If not found, the player will
lose one attempt.

\subsection{Tried letters}

\paragraph{} The string typed variable named \texttt{triedLetters} will
contains in order the letter already tried (with or without success) and each
time the user will input a character, the program will check that the letter
was not already tried, meaning that the letter is not contained in the
\texttt{triedLetters} variable. If it is contained, then the error message
\lstinline[style=prog]|"You already tried it!"| is displayed.

\subsection{Endgame}

\paragraph{} For the end game, two parameters are taken into account: the
hidden word variable and the remaining attempts variable. If the hidden word is
no more hidden, the player wins. Else if the remaining attempts is zero, then
the player looses.

\section{Details}

\subsection{Command line arguments}

\paragraph{} Command line arguments can be passed to the program. The
\texttt{--h} and \texttt{----help} options will make the program display an
help for the program command line usage and exit. The \texttt{--v} or
\texttt{----version} options will display the version number of the program and
exit. The \texttt{--d} or \texttt{----display} option takes one argument which
is the dictionary file containing the words to pick from. The header
\texttt{unistd.h} is used in order to be able to make use of the
\texttt{getopt} facility.

\subsection{Console clear}

\paragraph{} For each lines printed by the program each turn, the escape
sequence code \lstinline[style=prog]|"\033[A\033[2K"| is printed which will
	make the terminal cursor go one line up and any printed text after this
	escape sequence code will be printed over these lines.

\subsection{Invalid characters}

\paragraph{} Each time the player inputs a character, the character inputted is
checked through the function \lstinline[style=prog]|validLetter| and display
\lstinline[style=prog]|"Invalid character"| if the character inputted is not a
letter.  If the letter inputted is a \lstinline[style=prog]|'\n'|, the console
curser goes up one line more.

\subsection{\texttt{malloc} aversion}

\paragraph{} In order to prevent a \lstinline[style=prog]|malloc| from being
used in the function \lstinline[style=prog]|searchWord| of the
\texttt{utils/dict.c} file, a \lstinline[style=prog]|static| string is returned
and reset each time the function is called by putting the
\lstinline[style=prog]|'\0'| at the first character.

\subsection{Picture files checking and access}

\paragraph{} In case of non existing picture files, the program should still be
able to display default pictures. Thus, the program checks for the existence of
the picture files and if they are non existent, the default ones (hardcoded in
the \texttt{hardcodedPictures} variable) are used. If they exists, then the
program will display them and count how many lines they are in order to make
the terminal cursor go up as much as necessary. In order to acces a given
picture, the boolean value set by the function checking the existence of the
file is read, and depending of its value, the hardcoded values are used or the
files are read.

\chapter{The project}

\paragraph{} In order to generate good projects Makefiles and consistant
configurations files, the tools provided by \texttt{autconf} and
\texttt{automake} (sometimes combined with the name ``Autotools'' or the ``GNU
build system'') are used.

\section{Configuration files}

\subsection{\texttt{configure.ac}}

\paragraph{} The first file needed by the Autotools system is the
\texttt{configure.ac} file which will define rules that are project wide and
will check for the requested libraires, functions, etc\dots

\paragraph{} The first macro should be \texttt{AC_INIT} and take as parameter
the name of the project/package, the version of the package and the email to
report bug to. In order to use the \texttt{C99} version of \texttt{C}, the
macro \texttt{AC_PROG_CC_C99} is used. After that, the macro
\texttt{AC_CONFIG_HEADER} will define the file in which a list of
\lstinline[style=prog]|##define| used to get data about the project in source
files. Then a list of headers used to check the current system mainly for
headers. Afterwards, the \texttt{AC_CONFIG_FILES} macro will define which
Makefiles will be generated (therefore a \texttt{Makefile.am} of
\texttt{Makefile.in} is needed for each one of them). Finally, the macro
\texttt{AC_OUTPUT} will activate the generation of files.

\subsection{\texttt{Makefile.am}}

\paragraph{} There are several \texttt{Makefile.am} files in this project (3
precisely) and each one of them exists for its own reason and the reason is
mainly linked to the location of the \texttt{Makefile.am}.

\paragraph{} The first \texttt{Makefile.am} is the one at the root of the
project directory. It contains the description of the subdirectories and the
data files of the project.

\paragraph{} The second one is the \texttt{Makefile.am} in the \texttt{src}
directory. It contains personalized flags to be passed to the \texttt{gcc}
program at compilation time, the name of the generated binairies, the directory
in which the data files will be installed, the source files of the
corresponding binairies and the data files.

\paragraph{} The third \texttt{Makefile.am} is the one in the \texttt{doc}
directory. It basically contains the rules for making and cleaning \LaTeX
files.

\section{Generating files}

\subsection{\texttt{autoreconf}}

\paragraph{} The main reason for using the ``GNU build system'' is to generate
secondary files. The command used to generate a subset of these files is
\texttt{autoreconf --install}. It will basically generate the
\texttt{configure} script, the \texttt{config.h} template (called
\texttt{config.h.in}), \texttt{Makefile} templates (called
\texttt{Makefile.in}), and other subsidiary files.

\subsection{The \texttt{configure} script}

\paragraph{} After this step (which is not required when downloading this or
any other project using Autotools because the \texttt{configure} script and the
other subsidiary files are already included in the tarball), it is required to
run the \texttt{configure} script with optional arguments (like
\texttt{----prefix=/path} for specifying the installation path) which can all
be listed using \texttt{----help}. This will finally generate all the
\texttt{Makefile} and the \texttt{config.h} files necessary to the compilation
and installation of the program.

\subsection{Compilation and installation}

\paragraph{} The final step is then to run the commands \texttt{make} and
\texttt{make install} if the installation of the project is wanted. This will
compile the project and the executable binary can be found in the \texttt{src}
subdirectory.

\subsection{Tarball}

\paragraph{} The Autotools system provides a way to generate different kinds of
tarballs to share the project. The command is simply \texttt{make dist} or
\texttt{make dist-\textit{type}} for another kind of tarball i.e.  \texttt{make
dist-xz} for a \texttt{.tar.xz} tarball or \texttt{make dist-zip} for a
\texttt{.zip} archive.

\end{document}
% vim: spell
