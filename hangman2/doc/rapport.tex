\documentclass{report}

\usepackage[utf8]{inputenc}
\usepackage[T1]{fontenc}
\usepackage{textcomp}
\usepackage[french]{babel}
\usepackage[]{amsmath}
\usepackage{amsfonts}
\usepackage{mathrsfs}
\usepackage{mathtools}
\usepackage{mathabx}
\usepackage[]{graphicx}
\usepackage{fullpage}
\usepackage[usenames,dvipsnames]{color}
\usepackage{listingsutf8}
\usepackage[hidelinks]{hyperref}

\definecolor{gray}{gray}{0.70}

\lstset{%
	tabsize=4,
	basicstyle=\small\ttfamily,
	numberstyle=\scriptsize\ttfamily,
	texcl=true,
	extendedchars=true,
	inputencoding=utf8/latin1
}

\lstdefinestyle{prog}{%
	caption=Programme,
	numbers=left,
	frame=L,
	language=c,
	tabsize=4,
	basicstyle=\small\ttfamily,
	numberstyle=\scriptsize\ttfamily,
	keywordstyle=\color{BlueViolet},
	stringstyle=\color{Green},
	commentstyle=\color{gray},
	showstringspaces=false
}

\lstdefinestyle{output}{%
	caption=Résultat,
	frame=single,
	belowcaptionskip=1\baselineskip,
	breaklines=true
}

\lstdefinestyle{outfile}{%
	caption=Fichier de sortie,
	frame=shadowbox,
	rulesepcolor=\color{gray},
	belowcaptionskip=1\baselineskip,
	breaklines=true
}

\author{Rémi \textsc{Nicole}}
\title{Rapport du TP 7 d'IGI-2001}
\date{}

\begin{document}

\maketitle

\chapter{Exercice 1}
\lstinputlisting[style=prog]{exo1.c}
\lstinputlisting[style=output]{exo1.out}
\lstinputlisting[style=outfile]{exo1.f.out}

\chapter{Exercice 2}
\lstinputlisting[style=prog]{exo2.c}
\lstinputlisting[style=output]{exo2.out}

\chapter{Exercice 3}
\lstinputlisting[style=prog]{exo3.c}
\lstinputlisting[style=output]{exo3.out}

\chapter{Exercice 4}
\lstinputlisting[style=prog]{exo4.c}
\pagebreak
\begin{figure}[h]
	\centering
	\includegraphics{{exo4.out}.png}
	\caption{Capture d'écran du résultat de l'exercice 4 avec initialisation
	aléatoire}
\end{figure}

\section{Génération aléatoire}

\paragraph{} Pour la génération aléatoire, il a fallu utiliser une probabilité
pondérée car il est considéré que un pixel non noir est ``vivant''. Ainsi, un
pixel a une probabilité de $\dfrac{1}{5}$ d'être ``vivant'' et chaque canal
correspondant à soit la couleur rouge, verte ou bleue est initialisée
aléatoirement (entre \texttt{0x0} et \texttt{0xFF}, soit dans l'intervalle
$\left\ldbrack0 ; 256\right\ldbrack$). On utilise donc la fonction
\lstinline[style=prog]|srand(time(NULL))| pour initialiser la séquence de
nombre pseudo aléatoires avec le nombre de secondes écoulées depuis 1970 et on
utilise \lstinline[style=prog]|rand() % 256| pour obtenir un nombre aléatoire
dans l'intervalle $\left\ldbrack0 ; 256\right\ldbrack$.

\section{Calcul du lendemain}

\paragraph{} Pour le calcul du ``jour d'après'', il a fallu créer un tableau
temporaire contenant le résultat avant de le copier entièrement (avec
\lstinline[style=prog]|memcopy|) dans le tableau original pour éviter que les
``cellules'' qui étaient vivante le jour d'avant et sont devenue mortes ne
compte pas comme voisin et inversement.

\paragraph{} De plus, en raison de la gestion de couleur qui a été ajouté dans
le programme, il a fallu déterminer la couleur des ``cellules naissantes''.
Ainsi, une fonction \lstinline[style=prog]|mixNeighbousColors| a été crée afin
de faire un ``mélange'' des couleurs des voisins de la cellule naissante. Pour
ce faire, il a suffit de récupérer chaque canaux de couleurs (autre que
l'alpha), de faire une moyenne de ces canaux des 3 ``voisins'' et de les
rassembler faisant ainsi une nouvelle couleur qui est un mélange de ses
``parents''.

\section{Améliorations}

\subsection{Utilisation de la ligne de commande}

\paragraph{} Afin d'avoir le contrôle de l'exécution du programme sans avoir à
le recompiler, la ligne de commande a été utilisée. Ainsi, le programme prends
un paramètre : soit l'option \texttt{----random} qui génèrera une image à
couleurs aléatoire de taille définie par les \lstinline[style=prog]|#define|
dans le code source à moins que 2 chiffres correspondant respectivement à la
largeur et à la hauteur de la fenêtre ne soient donnés après l'option
\texttt{----random}, soit l'option \texttt{----bmp} suivit d'un chemin
(relatif ou absolu) d'une image sous le format BMP qui sera analysée et
utilisée comme configuration initiale pour le Jeu de la Vie ou encore l'option
\texttt{----rle} suivit d'un chemin d'un fichier \texttt{.rle} contenant des
données RLE d'un motif du Jeu de la Vie (cf.~\ref{ssec-rle}). Il est aussi
possible d'activer le monde toroïdal (cf.~\ref{ssec-tor}) en ajoutant l'option
\texttt{----toroidal} à la fin de la ligne de commande.

\subsection{Analyse d'un fichier BMP}

\paragraph{} Afin d'analyser un fichier BMP, il a fallu passer par certaines
étapes, notamment la récupération de la taille, la détection du nombre d'octets
par pixel, et autres.\footnote{voir
\href{http://en.wikipedia.org/wiki/BMP_file_format}{l'article Wikipedia}}

\paragraph{} On notera tout de même que le premier pixel du tableau de pixel de
l'image est le pixel en bas à gauche de la représentation graphique de l'image.

\subsubsection{Vérification magique du fichier BMP}

\paragraph{} La première chose faite après vérification de la lisibilité du
fichier lorsqu'un paramètre autre que l'option \texttt{----random} est passé au
programme est vérifier si le nom du fichier correspond bien à un fichier BMP,
non pas au niveau de l'extension du fichier, mais au niveau des nombres
magiques. En effet, les deux premiers octets d'un fichier correspondant à une
image BMP sont \texttt{0x42} et \texttt{0x4D}, ce qui correspond au code ASCII
des lettres ``B'' et ``M''. C'est ce que font les lignes 399--408 du programme.

\subsubsection{Récupération de la largeur de hauteur de l'image}

\paragraph{} Afin de récupérer la largeur et la hauteur de l'image, les
$4\times2$ octets aux offsets \texttt{0x12} et \texttt{0x16} contenant
respectivement la largeur et la hauteur de l'image sont récupérés. Ainsi, si
les l'octet d'offset de \texttt{0x12} à \texttt{0x15} inclus il est contenu
\texttt{80 02 00 00} soit \texttt{00 00 02 80} en big-endian, ce la signifie
que l'image est de largeur \texttt{0x280} soit 640.

\subsubsection{Début et fin du tableau de pixel}

\paragraph{} Le début du tableau de pixel dans le fichier binaire du BMP est à
un offset qui est stocké l'offset \texttt{0xA}. Il est donc récupéré dans la
variable \lstinline[style=prog]|unsigned int pixelStart|. Afin de récupérer la
fin du tableau de pixel, la fonction \lstinline[style=prog]|fseek| est utilisée
pour aller à la fin du fichier et grâce à la fonction
\lstinline[style=prog]|ftell|, l'offset de fin de fichier est récupéré et
stocké dans la variabel \lstinline[style=prog]|unsigned int pixelEnd|.

\subsubsection{Nombres d'octets par pixel}

\paragraph{} Le nombre de bits par pixel peut varier en fonction de l'image (et
notamment à cause du canal alpha). Pour palier à cela, cette valeur est stockée
à l'offset \texttt{0x1C}. Elle est ensuite convertie en octets pour être
exploité plus directement.

\paragraph{} Dans le cas où il y a 3 octets par pixel, il n'y a pas de canal
alpha et l'ordre des canaux de couleurs est bleu, vert et ensuite rouge, de par
le fait que les données sont stockées en little-endian.

\paragraph{} Dans le cas où il y a 4 octets par pixel, le canal alpha est
présent et l'ordre des canaux change en fonction du logiciel ou de la personne
qui a créé le bitmap. Afin de connaître l'ordre des canaux, des masques sont
fournis de l'offset \texttt{0x36} (masque du canal rouge) à l'offset
\texttt{0x42} (masque du canal alpha), bien sûr de taille 4 octets et stocké en
little-endian.

\paragraph{} Afin d'utiliser le même code pour les images possédant 3 octets
par pixel et 4 octets par pixel, les variables
\lstinline[style=prog]|unsigned short int red|
\lstinline[style=prog]|unsigned short int green| et
\lstinline[style=prog]|unsigned short int blue| sont utilisées et contiennent
un chiffre correspondant à leur place dans les 3 ou 4 octets du pixel. Pour
calculer leur place dans les octets du pixel, il faut récupérer la place du
\texttt{0xFF} dans le masque parmi les autres \texttt{0x00}. On divise donc le
masque par \texttt{0xFF} ou $255$ et il n'y a plus qu'a obtenir le nombre de
fois que cette valeur a été multiplié par \texttt{0x100} soit $16\times16=256$.
On fera donc $\log_{256}\left(\dfrac{masque}{255}\right) =
\dfrac{\log\left(\dfrac{masque}{255}\right)}{\log\left(256\right)}$

\subsubsection{Calcul du padding du tableau de pixel}

\paragraph{} La partie tableau de pixels du fichier de l'image BMP possède un
``padding'' qui fait en sorte que à chaque fin de ligne de l'image, des données
sont ajoutées afin que le nombre d'octets de chaque ligne soient un multiple de
4. On a donc :

\begin{equation}
		padding =	\begin{dcases*}
						0 & si $nbOctetsParLigne \equiv 0[4]$\\
						4 - (nbOctetsParLigne \mod 4) & sinon
					\end{dcases*}
\end{equation}

Ce qui est équivalent à :

\begin{equation}
	padding = \big(4 - (tailleLigne * nbOctetsParPixels \mod 4)\big) \mod 4
\end{equation}

\paragraph{} Ainsi, à chaque fin de ligne, il faut ``sauter'' le padding pour
pouvoir accéder à la ligne suivante grâce à la ligne
\lstinline[style=prog]|fseek(fichier, padding, SEEK_CUR)|.

\subsection{Analyse d'un fichier RLE}\label{ssec-rle}

\paragraph{} Afin de pouvoir rentrer des motifs de manière plus simple que de
rentrer chaque pixel un par un ou d'essayer le mode aléatoire de manière
continue, un fichier \texttt{.rle} peut être fournis et va être analysé par le
programme.\footnote{Voir
\href{http://conwaylife.com/wiki/Run_Length_Encoded}{l'article sur ConwayLife}}

\subsubsection{Commentaires}

\paragraph{} Il y a 5 types de commentaires dans les fichiers \texttt{.rle} et
sont donc ignorés:

\begin{table}[h]
	\centering
	\begin{tabular}{|c|p{10cm}|}
		\hline
		\textbf{Commentaire} & \textbf{Description}\\
		\hline
		\#C ou \#c & Commentaire classique\\
		\hline
		\#N & Nom du motif\\
		\hline
		\#O & Auteur du motif\\
		\hline
		\#P ou \#R & Coordonnées du coin haut gauche
					(ignorées car le motif est placé au centre de la fenêtre)\\
		\hline
		\#r & Fournis les règles du Jeu de la Vie pour ce motif (ignorées)\\
		\hline
	\end{tabular}
	\caption{Formes de commentaires d'un fichier \texttt{.rle}}
	\label{tab:rlecomments}
\end{table}

\subsubsection{Largeur et hauteur du motif}

\paragraph{} La largeur et la hauteur du motif est généralement stocké après
les commentaires dans un ligne de type :

\texttt{x = m, y = n}

Cette ligne est donc lue par le programme et il en est déduit après la position
du motif dans la fenêtre en calculant la coordonnées de la cellule haut gauche
grâce à la formule :

\begin{equation}
	\begin{pmatrix}
		x\\y
	\end{pmatrix}
	=
	\begin{pmatrix}
		\dfrac{LargeurFenêtre}{2} - \dfrac{LargeurMotif}{2}\\
		\dfrac{HauteurFenêtre}{2} - \dfrac{HauteurMotif}{2}
	\end{pmatrix}
\end{equation}

\subsubsection{Données des ``cellules''}

\paragraph{} Généralement après les commentaires et obligatoirement après la
taille du motif, les données concernant les ``cellules'' sont stockées et
interprétables grâce à trois symboles et des chiffres :

\begin{table}[h]
	\centering
	\begin{tabular}{|c|p{5cm}|}
		\hline
		\textbf{Indicateur} & \textbf{Description}\\
		\hline
		b & Cellule vivante\\
		\hline
		o & Cellule morte\\
		\hline
		\$ & Fin de la ligne (sous entendu le reste des cellules de la ligne
								sont des cellules mortes)\\
		\hline
	\end{tabular}
	\caption{Identificateur de vie des cellules dans un fichier \texttt{.rle}}
	\label{tab:rletags}
\end{table}

\paragraph{} Chacun des identificateurs du tableau~\ref{tab:rletags} peuvent
prendre un compteur comme argument (qui sera placé avant le dit indicateur).
Cela impliquera que l'indicateur sera répété autant de fois qu'indiqué par le
compteur.

\paragraph{} La fin des données du fichier \texttt{.rle} est indiqué par le
symbole ``!'' qui sous entend comme le symbole ``\$'' que le reste des
``cellules'' sont des ``cellules mortes''.

\subsection{Allocation des couleurs}\label{ssec-col}

\paragraph{} Afin de ne pas allouer à nouveau de la mémoire pour des couleurs
ayant de la mémoire déjà allouée, la variable globale
\lstinline[style=prog]|allocatedColors| est utilisée. Il s'agit d'un tableau de
structure contenant la couleur (de type \lstinline[style=prog]|XColor|) et un
chiffre, 1 si la couleur a déjà de la mémoire allouée, 0 sinon. Le tableau est
trié de telle manière que l'élément n°$n \in \left\ldbrack0 ;
256^3\right\ldbrack$ soit la couleur \texttt{\#XXXXXX} avec \texttt{XXXXXX} la
représentation de $n$ en base $16$.

\subsection{Monde toroïdal}\label{ssec-tor}

\paragraph{} Puisque le programme ne supporte pas le fait d'avoir un monde
infini ou extensible à l'infini, il a été décidé que toutes cellules hors du
tableau sont considérées comme ``mortes''. Cependant, une alternative est
possible : il s'agit de considérer le tableau à deux dimensions contenant les
cellules (ou le monde) comme toroïdal, c'est-à-dire que les bords du tableau
sont connectées. Ainsi, une cellule située à la bordure droite du monde peut
influencer une cellule à la bordure gauche. Afin d'arriver à ces fins, des
conditions ont été rajoutées dans la fonction du calcul du nombre de voisins et
dans la fonction calculant le mélange des couleurs des cellules voisines. Afin
d'activer ce type de monde, l'option ligne de commande \texttt{----toroidal} a
été rajoutée.

\section{Bugs / améliorations possibles}

\subsection{Performances}

\paragraph{} Malgré le fait que les couleurs ne soient allouées qu'une seule
fois (cf.~\ref{ssec-col}) l'allocation des couleurs prends beaucoup de temps,
impliquant que le fait d'allouer une couleur uniquement lorsqu'elle est
nouvelle au moment de l'affichage fait baisser les performances du programmes.

\paragraph{} Une autre manière d'améliorer les performances du programme serait
d'utiliser le multi-threading pour le calcul du ``lendemain'' en divisant le
tableau en plusieurs parties. Cependant, cette amélioration pourrait se trouver
coûteuse en mémoire : en effet, afin que les tableaux résultant de chaque
thread n'influence pas les threads toujours en train de traiter le tableau, il
faudrait faire une copie du tableau original pour chaque thread. Un autre
manière de contrer ce problème serait de recalculer le lendemain pour les
parties critiques. En effet, puisque le traitement tableau est divisé, la
``mauvaise influence'' des autres threads n'est possible qu'au bordures de ces
divisions (d'autant plus si le monde est toroïdal).

\subsection{Monde toroïdal}

\paragraph{} Un bug est présent lorsque l'option \texttt{----toroidal} est
activé, il s'agit du mauvais calcul du mélange des couleurs des voisins
lorsqu'une ``cellule naissante'' naît sur une bordure avec des parents de
l'autre côté de la bordure. Biens que le calcul du mélange ne soit pas correct,
les couleurs restent cohérentes entre elles (elles ne sont pas pour autant
aléatoires).

\subsection{Format RLE}

\paragraph{} L'implémentation du format RLE dans le programme supporte la
taille du motif. Cependant, lorsque la taille du motif est supérieure à la
taille de la fenêtre par défaut ($200\times200$), rien n'est affiché.

\paragraph{} Certaines fonctionnalités du format RLE, comme les définitions des
règles pour le jeu de la vie, ou encore les coordonnées du coins haut gauche,
ne sont pas implémentées.

\subsection{Ligne de commande}

\paragraph{} Les options de la lignes de commandes ne sont pas échangeables
avec l'option \texttt{----toroidal}. Par exemple, la ligne \texttt{./exo4.bin
----toroidal ----random} ne marchera pas.

\subsection{Boucle infinie}

\paragraph{} La boucle infinie utilisée n'est pas pseudo-infinie, impliquant
que pour quitter le programme, il faut soit lui envoyer un signal
\texttt{SIGINT}, \texttt{SIGTERM} ou \texttt{SIGKILL} ou encore fermer la
fenêtre affichant le jeu de la vie, ce qui fera quitter le programme avec une
erreur. Une manière de palier à cela serait d'écouter des XEvents et d'attendre
l'appui d'une touche particulière (par exemple \texttt{q}) ou encore de gérer
la fermeture de la fenêtre de manière plus orthodoxe.

\subsection{Affichage}

\paragraph{} Il arrive que la librairie X11 refuse d'exécuter une requête (ce
qui se voit avec le message d'erreur lorsque la fenêtre X11 est fermée), ce qui
aura pour conséquence des pixels qui auront leur état inchangé. Par exemple,
une cellule vivante auparavant qui doit mourir sera toujours affichée comme
vivante si X11 refuse d'accéder à la requête et inversement. Il faudrait donc
trouver un moyen de détecter si X11 a bien exécuté la requête en cours et si
non, la réitérer.

\paragraph{} Au niveau de la fréquence d'images par secondes (fps), elle est
déterminée par la performance du programme. Cela implique que la vitesse de
changement d'état n'est pas forcément constante et c'est d'autant plus vrai car
le nombre de couleurs déjà allouées augmente considérablement au cours du
programme, augmentent par la suite le nombre d'images par seconde au cours du
temps. Afin de régler ce problème, l'utilisation de
\lstinline[style=prog]|nanosleep| est possible (bien plus précise que
\lstinline[style=prog]|sleep|) ou encore mieux, faire une différence de
timestamp en millisecondes (avec la structure retournée par
\lstinline[style=prog]|gettimeofday|) pour assurer une fréquence d'image
constante quand bien même avec des calculs de durée différente.

\end{document}
% vim: spelllang=fr:spell
