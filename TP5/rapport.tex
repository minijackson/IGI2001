\documentclass{report}

\usepackage[utf8]{inputenc}
\usepackage[T1]{fontenc}
\usepackage[french]{babel}
\usepackage[usenames,dvipsnames]{color}
\usepackage{listings}

\definecolor{gray}{gray}{0.70}

\lstset{%
	tabsize=4,
	basicstyle=\small\ttfamily,
	numberstyle=\scriptsize\ttfamily
}

\lstdefinestyle{prog}{%
	caption=Programme,
	numbers=left,
	frame=L,
	language=c,
	tabsize=4,
	basicstyle=\small\ttfamily,
	numberstyle=\scriptsize\ttfamily,
	keywordstyle=\color{BlueViolet},
	stringstyle=\color{Green},
	commentstyle=\color{gray},
	showstringspaces=false
}

\lstdefinestyle{output}{%
	caption=Résultat,
	frame=single,
	belowcaptionskip=1\baselineskip,
	breaklines=true
}

\author{Rémi \textsc{Nicole}}
\title{Rapport du TP 5 d'IGI-2001}
\date{}

\begin{document}

\maketitle

\chapter{Exercice 1}
\lstinputlisting[style=prog]{exo1.c}
\lstinputlisting[style=output]{out1}

\chapter{Exercice 2}
\lstinputlisting[style=prog]{exo2.c}
\lstinputlisting[style=output]{out2}

\section{Question 1}
Il existe un spécificateur de conversion qui affiche une variable en hexadécimal
(soit \texttt{x} ou \texttt{X}) mais il n'existe pas de spécificateur de conversion pour
afficher une variable en binaire.

\chapter{Exercice 3}
\lstinputlisting[style=prog]{exo3.c}
\lstinputlisting[style=output]{out3}

\chapter{Exercice 4}
\lstinputlisting[style=prog]{exo4.c}
\lstinputlisting[style=output]{out4}

\end{document}
