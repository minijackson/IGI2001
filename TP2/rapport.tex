\documentclass{report}

\usepackage[french]{babel}
\usepackage[utf8]{inputenc}
\usepackage[T1]{fontenc}
\usepackage{csvtools}

\begin{document}

\section{Exercice 1}

\begin{table}[h]
	\begin{tabular}{|l|l|c|c|}
		\hline
		\textbf{Type}	&\textbf{Format de printf}	&\textbf{Taille en octet}	&\textbf{Taille en bits}\\
		\hline
		char			&\%d						&1							&8	\\
		\hline
		unsigned char	&\%u						&1							&8	\\
		\hline
		short			&\%d						&2							&16	\\
		\hline
		unsigned short	&\%u						&2							&16	\\
		\hline
		int				&\%d						&4							&32	\\
		\hline
		unsigned int	&\%u						&4							&32	\\
		\hline
		long			&\%d						&10							&80	\\
		\hline
		long long		&\%lld (C99)				&10							&80	\\
		\hline
		float			&\%f ; \%g ; \%e			&4							&32	\\
		\hline
		double			&\%f ; \%g ; \%e			&10							&80 \\
		\hline
	\end{tabular}
	\caption{Types.csv}
	\label{tab:types}
\end{table}

\paragraph{}

On remarque au niveau des adresses qu'elles sont codées en hexadécimal et que
les adresses ne sont pas collées entre les variables.

\begin{table}[h]
	\centering
	\begin{tabular}{|l|c|c|c|}
		\hline
		\textbf{Type}	&\textbf{Adresse début}	&\textbf{Taille}	&\textbf{Adresse fin}	\\
		\hline
		char			&0x7fff4fb30490			&8					&0x7fff4fb30498	\\
		\hline
		unsigned char	&0x7fff4fb304a0			&8					&0x7fff4fb304a8	\\
		\hline
		short			&0x7fff4fb304b0			&16					&0x7fff4fb304c0	\\
		\hline
		unsigned short	&0x7fff4fb304c0			&16					&0x7fff4fb304d0	\\
		\hline
		int				&0x7fff4fb304d0			&32					&0x7fff4fb304f0	\\
		\hline
		unsigned int	&0x7fff4fb304e0			&32					&0x7fff4fb30500	\\
		\hline
		long			&0x7fff4fb30500			&80					&0x7fff4fb30550	\\
		\hline
		long long		&0x7fff4fb30510			&80					&0x7fff4fb30560	\\
		\hline
		float			&0x7fff4fb304f0			&32					&0x7fff4fb30510	\\
		\hline
		double			&0x7fff4fb30520			&80					&0x7fff4fb30570	\\
		\hline
	\end{tabular}
	\caption{Adresses.csv}
	\label{tab:addr}
\end{table}

\section{Exercice 2}

Le fait que les factoriels calculés soient incorrects est dû au fait qu'il y
ait un dépassement de capacité. Afin de palier à cela, le \texttt{long long
unsigned int} a été utilisé. Après modifications, le factoriel le plus élevé
calculé correctement est le factoriel 20. Afin d'afficher les factoriels de 0 à
100, il suffit d'exécuter la commande \texttt{./exo2 \{0..100\}} où la partie
\texttt{\{0..100\}} est interprétée par le Shell.

\end{document}
